\documentclass{university}

\begin{document}
\setupdocument

\section{}

\subsection{}
\subsubsection{}
\begin{enumerate}
    \item \lr{Performance measure} :
    \begin{enumerate}
        \item رعایت قوانین رانندگی
        \item واکنش مناسب هنگام خطر
        \item انتخاب بهترین و سریع‌ترین مسیر
        \item ایمنی در رانندگی
    \end{enumerate}
    \item \lr{Environment} :
    تمامی خیابان‌ها و 
    جاده‌هایی که تاکسی باید در آن تردد کند. عوامل و موانع طبیعی موجود در جاده و راه‌ها. تابلوهای راهنمایی و رانندگی.
    \item \lr{Actuator} :
    عملکردهای خودرو مانند گاز، ترمز، حرکت 
    فرمان، راهنما، چراغ و \dots.
    \item \lr{Sensor} : 
    دوربین‌ها و سنسورهایی که بر بدنه خودرو نصب 
    شده‌اند. همچنین خودرو باید از مسافر نیز فرمان 
    بگیرد. سنسور 
    \lr{GPS} و 
    سنسورهای درونی خودرو نظیر دماسنج.

\end{enumerate}
\subsubsection{}
\begin{enumerate}
    \item \lr{Performance measure} : بردن بازی، زمان انجام هر حرکت، تابع قوانین بازی بودن
    \item \lr{Environment} : صفحه بازی و مهره‌ها 
    \item \lr{Actuator} : احتمالا یک بازو مکانیکی برای کاشتن مهره در صفحه بازی
    \item \lr{Sensor} : یک یا چند دوربین برای تشخیص جایگاه مهره‌ها موجود در صفحه بازی (اگر جای مهره‌ها را با پردازش تصویر تشخیص دهد. در غیر این صورت سنسورهای مناسب سیستم طراحی شده)
\end{enumerate}

\subsection{}
\begin{table}[!htbp]
\centering
\begin{tabular}{|c|c|}
    \hline
    \textbf{\lr{Tic-tac-toe robot}} & \textbf{\lr{Chess with time control}} \\
    \hline
    \lr{Fully observable} & \lr{Fully observable} \\
    \lr{Strategic} & \lr{Strategic} \\
    \lr{Sequential} & \lr{Sequential} \\
    \lr{Static} & \lr{Semidynamic {\footnotesize time control}} \\
    \lr{Multi agent} & \lr{Multi agent} \\
    \lr{Continuous {\footnotesize pieces placement}} & \lr{Discrete} \\
    \hline
\end{tabular}
\end{table}

\section{}

\subsection{حالت‌ها}
حالت‌ها جایگاه اعداد در مربع هستند. برای نمونه اگر مربع داده شده را به صورت 
یک آرایه دو بعدی در نظر بگیریم که 3 سطر و 3 ستون دارد، هر جایگشت اعداد 1 تا 9 در خانه‌های 
این آرایه، به صورتی که هر عدد تنها 1 بار ظاهر شود یک حالت برای این مسئله است.

\subsection{عملیات‌ها}
عملیات موجود در این مدلسازی، جابجایی دو خانه است. حال این جابجایی طبق محدودیت مسئله 
بین خانه‌ای که مقدار آن 9 است و یکی از خانه‌های مجاور(دارای ضلع مشترک) آن انجام می‌شود. 
برای سادگی می‌توان 4 عملیات زیر را در نظر گرفت.
\begin{enumerate}
    \item \lr{U} : مقدار خانه دارای مقدار 9 با مقدار خانه بالایی آن جابجا شود
    \item \lr{D} : مقدار خانه دارای مقدار 9 با مقدار خانه پائینی آن جابجا شود
    \item \lr{R} : مقدار خانه دارای مقدار 9 با مقدار خانه راستی آن جابجا شود
    \item \lr{L} : مقدار خانه دارای مقدار 9 با مقدار خانه چپی آن جابجا شود
\end{enumerate}
البته هر عملیات در صورتی معتبر است که خانه‌ای که مقدار آن 9 است در گوشه به گونه‌ای نباشد که خانه متناظر با 
عملیات وجود نداشته باشد. برای مثال برای خانه‌ی 
\lr{$(0, 0)$}
در مدلسازی با آرایه دو بعدی، عملیات‌های 
\lr{U}
و 
\lr{L} 
معتبر نیستند.

\subsection{شرط رسیدن به هدف}
شرط رسیدن به هدف طبق مطلوبات مسئله، برابر شدن جمع اعداد روی هر سطر، ستون و قطر است.

\section{}
\begin{gather}
    DFS(S,G1) : S \rightarrow A \rightarrow D \rightarrow C \rightarrow G1, cost = 12
\end{gather}

\end{document}