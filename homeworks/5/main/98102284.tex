\documentclass{university}

\course{هوش مصنوعی}
\subject{تمرین پنجم بخش اول}
\professor{دکتر رهبان}

\begin{document}

\setupdocument

\section{}
از آنجایی که 
\lr{$k$} 
استیت داریم، 
\lr{$k$} 
پارامتر برای نشان دادن استیت اولیه نیاز داریم. 

برای هر استیت 
\lr{$S_i$} 
با یک احتمالی به استیت 
\lr{$S_j$} 
می‌رویم که 
\lr{$1 \leq i, j \leq k$}
است. پس در اینجا نیز 
\lr{$k^2$}تا 
پارامتر داریم. 

هر مشاهده 
\lr{$E_i$} 
که داشته باشیم، با یک احتمال در استیت 
\lr{$S_j$}
هستیم که 
\lr{$1 \leq i \leq m$}
و
\lr{$1 \leq j \leq k$}
است. 
پس در اینجا هم 
\lr{$m\times k$} 
پارامتر داریم. 

$$
\text{\lr{Number of Parameters}} = k + k^2 + m \times k
$$

\subsection{}

در جدول 
\ref{tab:forward}
مراحل اجرای الگوریتم را مشاهده می‌کنیم. اگر سطرهای آخرین جدول را جمع بزنیم احتمال خواسته شده به دست می‌آید.

$$
0.12426400800000002 + 0.000950699 = 0.12521470700000004 \approx 0.125
$$

\begin{latin}
\begin{table}
    \centering
    \begin{tabular}{|c|c|}
        \hline 
        $S_1$ & $P(S_1, O_1 = 0)$ \\
        \hline 
        A & $0.99 \times 0.8 = 0.792$ \\
        B & $0.01 \times 0.1 = 0.001$ \\
        \hline 
    \end{tabular}
    \centering
    \begin{tabular}{|c|c|}
        \hline 
        $S_2$ & $P(S_2, O_1 = 0, O_2 = 1)$ \\
        \hline 
        A & $0.2 \times (0.99 \times 0.792 + 0.01 \times 0.001) = 0.156818$ \\
        B & $0.9 \times (0.01 \times 0.792 + 0.99 \times 0.001) = 0.008019$ \\
        \hline 
    \end{tabular}
    \centering
    \begin{tabular}{|c|c|}
        \hline 
        $S_3$ & $P(S_3, O_1 = 0, O_2 = 1, O_3 = 0)$ \\
        \hline 
        A & $0.8 \times (0.99 \times 0.156818 + 0.01 \times 0.008019) = 0.12426400800000002$ \\
        B & $0.1 \times (0.01 \times 0.156818 + 0.99 \times 0.008019) = 0.000950699$ \\
        \hline 
    \end{tabular}
    \caption{Forward algorithm}
    \label{tab:forward}
\end{table}
\end{latin}


\subsection{}
با توجه به جدول 
\ref{tab:viterbi}
که مراحل اجرای الگوریتم ویتربی را نشان می‌دهد، محتمل‌ترین دنباله در زیر آمده است. 
$$
S_1 = A, S_2 = A, S_3 = A
$$

\begin{latin}
\begin{table}
    \centering
    \begin{tabular}{|c|c|}
        \hline 
        $S_1$ & $P(S_1, O_1 = 0)$ \\
        \hline 
        A & $0.99 \times 0.8 = 0.792$ \\
        B & $0.01 \times 0.1 = 0.001$ \\
        \hline 
    \end{tabular}
    \centering
    \begin{tabular}{|c|c|c|}
        \hline 
        $S_2$ & $P(S_2, O_1 = 0, O_2 = 1)$ & arg max \\
        \hline 
        A & $0.2 \times \max(0.99 \times 0.792, 0.01 \times 0.001) = 0.156816$ & $S_1 = A$\\
        B & $0.9 \times \max(0.01 \times 0.792, 0.99 \times 0.001) = 0.007128$ & $S_1 = A$\\
        \hline 
    \end{tabular}
    \centering
    \begin{tabular}{|c|c|c|}
        \hline 
        $S_3$ & $P(S_3, O_1 = 0, O_2 = 1, O_3 = 0)$ & arg max \\
        \hline 
        A & $0.8 \times \max(0.99 \times 0.156816, 0.01 \times 0.007128) = 0.124198272$ & $S_2 = A$\\
        B & $0.1 \times \max(0.01 \times 0.156816, 0.99 \times 0.007128) = 0.000705672$ & $S_2 = B$\\
        \hline 
    \end{tabular}
    \caption{Viterbi algorithm}
    \label{tab:viterbi}
\end{table}
\end{latin}

\subsection{}
در این مثال خاص چون احتمال تغییر حالت از 
\lr{A} 
به 
\lr{B} 
و برعکس بسیار کم است، مجموع 
\lr{transsion probability}ها 
شبیه به ماکسیمم گرفتن عمل می‌کند. (ضریب یکی انقدر کوچک است که باعث می‌شود مجوع خیلی نزدیک به عدد دیگر باشد)
ولی این برابری در حالت کلی برقرار نیست. 
\end{document}