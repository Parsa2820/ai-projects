\documentclass{university}

\course{هوش مصنوعی}
\subject{سوالات نظری مینی‌پروژه اول}
\professor{دکتر رهبان}

\begin{document}

\setupdocument

\section{}
\subsection{}
\subsubsection{}
فضای حالت این مسئله، می‌تواند یک آرایه دو بعدی 
\lr{$M\times N$}
باشد که مقدار خانه‌های خالی آن 
\lr{False}
و مقدار خانه‌های دیوار آن 
\lr{True}
است. همچنین مکان حشره و خانه 
\lr{X}
هر کدام به صورت 
\lr{$(i, j)$}
که 
\lr{$0 \leq i \leq M-1$}
و
\lr{$0 \leq j \leq N-1$}
است، مشخص می‌شوند. 

\subsubsection{}
هر کدام از خانه‌های آرایه دو مقدار دارند.
مکان حشره و مکان خانه
\lr{X}
هر کدام 
\lr{$M\times N$}
مقدار دارند. پس در کل اندازه فضای حالت به صورت زیر محاسبه می‌شود. 
$$
2^{M \times N} + 2 \times (M \times N)
$$

\subsection{}
\subsubsection{}
همان فضای حالت بخش قبل است، با این تفاوت که به جای مکان خانه 
\lr{X}
مکان حشره دوم را نگه می‌داریم.

\subsubsection{}
هر کدام از خانه‌های آرایه دو مقدار دارند.
مکان دو حشره هر کدام 
\lr{$M\times N - W$}
مقدار دارند. که در آن 
\lr{$0 \leq W < M\times N - 2$}
تعداد دیوارها است. پس در کل اندازه فضای حالت به صورت زیر محاسبه می‌شود. 
$$
\Sigma_{w = 0}^{M \times N - 3} \binom{M \times N}{w} + 2 \times (M \times N - w)
$$

\section{}
\subsection{}
میخواهیم به هر 10 شهر برویم. پس هر کروموزوم 10 ژن دارد.

\subsection{}
الگوریتم 
\lr{crossover}
جدید را به این صورت تعریف می‌کنیم که یک عدد رندوم از 
\lr{$[0, 9]$}
انتخاب کند. سپس به تعداد این عدد رندوم عدد رندوم در همین بازه تولید کند. 
(\lr{$R_1, R_2, \dots, R_n$})

حال به ازای هر 
\lr{$R_i$}،
\lr{$S_i$}
را شماره خانه‌ای از کروموزوم دوم در نظر می‌گیریم که مقدار آن با خانه 
\lr{$R_i$}ام 
از کروموزوم اول برابر باشد. سپس در هر دو کروموزوم جای خانه‌های 
\lr{$R_i$}
و
\lr{$S_i$}
را عوض می‌کنیم.

\subsection{}
الگوریتم 
\lr{mutation}
را به این صورت تغییر می‌دهیم که یک عدد رندوم از 
\lr{$[0, 9]$}
انتخاب کند.
(\lr{$i$}) 
سپس یک شهر رندوم انتخاب کند.
(\lr{$x$})
مقدار 
\lr{$x$}
در کروموزوم را به مقدار خانه 
\lr{$i$}ام 
از همان کروموزوم تغییر می‌دهیم. سپس مقدار خانه 
\lr{$i$}ام 
را برابر 
\lr{$x$} 
قرار می‌دهیم. اینگونه با حفط سازگاری عملیات جهش را انجام داده‌ایم. 

\section{}
\subsection{}
\newcommand{\f}[6]{
    \ADD{#1}{#2}{\numone}
    \MULTIPLY{2}{#3}{\numtwo}
    \ADD{\numone}{\numtwo}{\numthree}
    \SUBTRACT{\numthree}{#4}{\numfour}
    \SUBTRACT{\numfour}{#5}{\numfive}
    \ADD{\numfive}{#6}{\numsix}
    \numsix
}
\begin{gather*}
    f(x_1) = \f{7}{6}{5}{3}{8}{4} \\
    f(x_2) = \f{9}{0}{3}{6}{4}{2} \\
    f(x_3) = \f{9}{2}{8}{3}{1}{3} \\
    f(x_4) = \f{2}{3}{2}{3}{8}{4}
\end{gather*}

\subsection{}
\subsubsection{}
فیت‌ترین کروموزوم‌ها، 
\lr{$x_1$}
و 
\lr{$x_3$}
هستند. 
\begin{gather*}
    \begin{drcases}
        x_1 = {\color{red}765} {\color{blue}384} \\
        x_3 = {\color{blue}928} {\color{red}313}
    \end{drcases} \rightarrow
    \begin{cases}
        x_5 = {\color{red}765} {\color{red}313} \\
        x_6 = {\color{blue}928} {\color{blue}384}
    \end{cases}
\end{gather*}

\subsubsection{}
دو غیر فیت‌ترین کروموزوم‌ها، 
\lr{$x_2$}
و 
\lr{$x_4$}
هستند. 
\begin{gather*}
    \begin{drcases}
        x_2 = {\color{red}90} {\color{blue}36} {\color{red}42} \\
        x_4 = {\color{blue}23} {\color{red}23} {\color{blue}84}
    \end{drcases} \rightarrow
    \begin{cases}
        x_7 = {\color{red}90} {\color{red}23} {\color{red}42} \\
        x_8 = {\color{blue}23} {\color{blue}36} {\color{blue}84}
    \end{cases}
\end{gather*}

\subsection{}
\begin{gather*}
    f(x_5) = \f{7}{6}{5}{3}{1}{3} \\
    f(x_6) = \f{9}{2}{8}{3}{8}{4} \\
    f(x_7) = \f{9}{0}{2}{3}{4}{2} \\
    f(x_8) = \f{2}{3}{3}{6}{8}{4}
\end{gather*}

\subsection{}
بین کروموزوم‌های موجود، کروموزوم بهینه همچنان کروموزوم 
\lr{$x_3$}
است. بین کروموزوم‌های نسل جدید کروموزوم 
\lr{$x_5$}
فیت‌تر از بقیه است.

\subsection{}
می‌دانیم بهینه‌ترین کروموزوم، 
\lr{$x_f = 999009$}
است. این رشته دو مقدار 0 و چهار مقدار 9 دارد. با توجه به اینکه کروموزوم‌های موجود در مجموع 
دو مقدار 9 و یک مقدار صفر دارند، تنها با 
\lr{crossover} 
هیچگاه به بهینه‌ترین کروموزوم نمی‌رسیم. زیرا با 
\lr{crossover}
تنها ارقام جابجا می‌شوند ولی تعدادشان کم و زیاد نمی‌شود. پس نیاز به 
\lr{mutation}
داریم تا تعداد ارقام موجود تغییر کند. 
\end{document}