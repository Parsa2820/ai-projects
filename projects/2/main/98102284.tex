\documentclass{university}

\course{هوش مصنوعی}
\subject{سوالات نظری مینی پروژه دوم}
\professor{دکتر رهبان}

\begin{document}

\setupdocument

\section{}


\section{}
\subsection{}
میدانیم اگر گراف قیود به شکل درخت باشد، اگر از ترتیب 
\lr{topological sort} 
برای مقداردهی استفاده کنیم، نیاز به 
\lr{backtrack} 
نخواهیم داشت. 
\footnote{این قضیه در کلاس مطرح شده است}
پس ترتیب مورد نظر 
\lr{C-A-B-D-E-F} 
است. 

\end{document}