\documentclass[en]{university}

\faculty{Department of Computer Engineering}
\course{Artificial Intelligence}
\subject{Mini Project 3 Theory Questions}
\professor{Dr. Rohban}
\student{Parsa Mohammadian}

\begin{document}

\setupdocument

\section{}

\subsection{}

We decide if a move violates game rules or not by using d-separation algroithm to find out 
if A and B are still independent. A valid move for each of the graphs is illustrated in 
figure \ref{fig:validmoves}.

\begin{figure}[!htbp]
    \centering
    \includegraphics[width=0.8\textwidth]{assets/1-a.png}
    \caption{Valid Moves}
    \label{fig:validmoves}
\end{figure}

\subsection{}

In a tree structure, I propose a move for first player and consider every possible move for second player and evaluate outcome 
from the leaves. If I can find a tree with all win outcomes, then I find a win strategy for first player. 

First graph tree is shown in figure \ref{fig:tree1}. So this graph has a win strategy for first player.

Second graph tree is shown in figure \ref{fig:tree2}. So this graph has a win strategy for first player too.

\begin{figure}[!htbp]
    \centering
    \includegraphics[width=\textwidth]{assets/1-b-1.png}
    \caption{First Graph Tree}
    \label{fig:tree1}
\end{figure}

\begin{figure}[!htbp]
    \centering
    \includegraphics[width=\textwidth]{assets/1-b-2.png}
    \caption{Second Graph Tree}
    \label{fig:tree2}
\end{figure}

\section{}

Base factor headers are:

\begin{align*}
    A : \begin{bmatrix}
        A & C & D & P(A | C, D)
    \end{bmatrix} \\
    B : \begin{bmatrix}
        B & D & E & G & P(B | D, E, G)
    \end{bmatrix} \\
    C : \begin{bmatrix}
        C & F & I & P(C | F, I)
    \end{bmatrix} \\
    D : \begin{bmatrix}
        D & G & H & P(D | G, H)
    \end{bmatrix} \\
    E : \begin{bmatrix}
        E & P(E)
    \end{bmatrix} \\
    F : \begin{bmatrix}
        F & H & P(F | H)
    \end{bmatrix} \\
    G : \begin{bmatrix}
        G & H & P(G | H)
    \end{bmatrix} \\
    H : \begin{bmatrix}
        H & I & P(H | I)
    \end{bmatrix} \\
    I : \begin{bmatrix}
        I & P(I)
    \end{bmatrix}
\end{align*}

\subsection{B, E, D, C, H, I}

\begin{align*}
    f_1 = \text{Join} (B) \rightarrow \text{Eliminate} (B) : \begin{bmatrix}
        D & E & G & P(D, E, G)
    \end{bmatrix} \\
    f_2 = \text{Join} (f_1, E) \rightarrow \text{Eliminate} (E) : \begin{bmatrix}
        D & G & P(D, G)
    \end{bmatrix} \\
    f_3 = \text{Join} (f_2, A, D) \rightarrow \text{Eliminate} (D) : \begin{bmatrix}
        A & C & G & H & P(A, C, G, H)
    \end{bmatrix} \\
    f_4 = \text{Join} (f_3, C) \rightarrow \text{Eliminate} (C) : \begin{bmatrix}
        A & G & H & F & I & P(A, G, H, F, I)
    \end{bmatrix} \\
    f_5 = \text{Join} (f_4, F, G, H) \rightarrow \text{Eliminate} (H) : \begin{bmatrix}
        A & F & G & I & P(A, F, G, I)
    \end{bmatrix} \\
    f_6 = \text{Join} (f_4, I) \rightarrow \text{Eliminate} (I) : \begin{bmatrix}
        A & G & F & P(A, G, F)
    \end{bmatrix}
\end{align*}

\subsection{I, H, C, D, E, B}

\begin{align*}
    f_1 = \text{Join} (C, H, I) \rightarrow \text{Eliminate} (I) : \begin{bmatrix}
        C & F & H & P(C, F, H)
    \end{bmatrix} \\
    f_2 = \text{Join} (f_1, D, F, G) \rightarrow \text{Eliminate} (H) : \begin{bmatrix}
        C & F & D & G & P(C, F, D, G)
    \end{bmatrix} \\
    f_3 = \text{Join} (f_2, A) \rightarrow \text{Eliminate} (C) : \begin{bmatrix}
        A & D & F & G & P(A, D, F, G)
    \end{bmatrix} \\
    f_4 = \text{Join} (f_3, B) \rightarrow \text{Eliminate} (D) : \begin{bmatrix}
        A & F & G & B & E & P(A, F, G, B, E)
    \end{bmatrix} \\
    f_5 = \text{Join} (f_4, E) \rightarrow \text{Eliminate} (E) : \begin{bmatrix}
        A & F & G & B & P(A, F, G, B)
    \end{bmatrix} \\
    f_6 = \text{Join} (f_4) \rightarrow \text{Eliminate} (B) : \begin{bmatrix}
        A & F & G & P(A, F, G)
    \end{bmatrix}
\end{align*}

\subsection{Comparison}

The maximum size of the factor is $2^5$ for both orderings. So we should compare each factor size. Clearly 
the first order is better one and need less time to be computed.

\section{}

\subsection{}

\begin{gather*}
    T : \text{Today} \\
    Y : \text{Yesterday} \\
    P(T = D) = P(T = D | Y = D) \times P(Y = D) \\ + P(T = D | Y = N) \times P(Y = N) \\
    \text{$Y$ and $T$ have the same distribution} \\
    \Rightarrow P(T = D) = 0.3 \times P(T = D) + 0.1 \times P(T = N) \\
    P(T = N) = 1 - P(T = D) \\
    \Rightarrow P(T = D) = 0.3 \times P(T = D) + 0.1 \times (1 - P(T = D)) \\
    = 0.3 \times P(T = D) + 0.1 - 0.1 \times P(T = D) \\
    = 0.2 \times P(T = D) + 0.1 \\
    \Rightarrow 0.8 \times P(T = D) = 0.1 \Rightarrow P(T = D) = \frac{0.1}{0.8} = 0.125 \\
\end{gather*}

\subsection{}

\begin{gather*}
    P(X = x | Dizinnes, Shortness of Breath) \\ \propto P(X = x, Dizinnes, Shortness of Breath) \\
    = \sum_{y} P(X = x) \times P(Shortness of Breath | X = x) \\ \times P(Dizinnes | Y) \times P(Y | X = x) \\
    \Rightarrow P(X = x | Dizinnes, Shortness of Breath) \\ = \frac{P(X = x, Dizinnes, Shortness of Breath)}{\sum_{x} P(X = x, Dizinnes, Shortness of Breath)} \\
    \Rightarrow P(X = Sepsis, Dizinnes, Shortness of Breath) \\ = 0.003 \times 0.85 \times 0.8 \times 0.7 + 0.003 \times 0.85 \times 0.7 \times 0.5 = 0.0023205 \\
    \Rightarrow P(X = Heart Muscle Weakness, Dizinnes, Shortness of Breath) \\ = 0.02 \times 0.5 \times 0.8 \times 0.7 + 0.02 \times 0.5 \times 0.7 \times 0.1 = 0.0063 \\
    \Rightarrow P(X = Sepsis | Dizinnes, Shortness of Breath) \\ = \frac{0.0023205}{0.0023205 + 0.0063} = 0.26918392204628505 \\
    \Rightarrow P(X = Heart Muscle Weakness | Dizinnes, Shortness of Breath) \\ = \frac{0.0063}{0.0023205 + 0.0063} = 0.7308160779537151
\end{gather*}

So the patient is more likely to have Heart Muscle Weakness.

\end{document}